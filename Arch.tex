\begin{frame}{Arch Linux}
    \begin{figure}[h]
        \includegraphics[width=0.3\textwidth]{images/Archlinux-icon-crystal-64.png}
    \end{figure}
    \textbf{Arch Linux} è una distribuzione GNU/Linux\\

    \begin{itemize}
        \item Creata da Judd Vinet nel 2002
        \item Basata sulla filosofia \textit{KISS}
        \item Inizialmente ispirata a \textit{CRUX Linux}
    \end{itemize}
\end{frame}

%-----------------------------------------------

\begin{frame}{KISS}
    \textbf{KISS} è un acronimo per \textbf{Keep it simple, stupid} che suggerisce di evitare soluzioni  complesse
    poiché queste possono creare problemi. La semplicità è la cosa migliore.\\

    Arch Linux è basato su questa filosofia e per questa ragione \sout{non troverai nessun installer o software per la configurazione}.\\
    Le configurazioni vengono gestite tramite la modifica di semplici file di testo(come papà UNIX vuole).
\end{frame}

%-----------------------------------------------

\begin{frame}{Pro Arch Linux}
    \begin{itemize}
        \item Rolling-release
        \item Applicativi per come sono stati pensati dagli sviluppatori
        \item Installazione in base alle tue esigenze
        \item Sai tutto quello che ci sta dentro perché l'hai installato tu
        \item Ottima per esplorare GNU/Linux e capire meglio il suo funzionamento
        \item Arch Linux Wiki e Community
        \item AUR(questione di opinioni...)
        \item \dots
        \item ...usare ``Btw I use Arch'' come intercalare
    \end{itemize}

\end{frame}

%-----------------------------------------------

\begin{frame}{Contro Arch Linux}
    \begin{itemize}
        \item Ostica per i neofiti
        \item Installazione non immediata
        \item Se non si fa attenzione potrebbero venir fuori dei problemi (Soluzione: snapshot di sistema)
    \end{itemize}
\end{frame}

%-----------------------------------------------

\begin{frame}{4 falsi miti su Arch Linux}
    \begin{itemize}
        \item Arch Linux è estremamente complicata
        \item È meno stabile/sicura perché è una rolling-release
        \item Un aggiornamento può rompere il sistema
        \item Gli utenti di Arch sono spocchiosi
    \end{itemize}
\end{frame}

%-----------------------------------------------

\begin{frame}{Arch Linux è estremamente complicata}
    \alert{Errato!} Perché?
    \begin{itemize}
        \item La difficoltà risiede principalmente nella sua installazione e configurazione
        \item Dopo aver pagato il costo di apprendimento, si è più produttivi
        \item Installazioni e configurazioni articolate sono più semplici da realizzare
        
    \end{itemize}
\end{frame}

%-----------------------------------------------

\begin{frame}{È meno stabile/sicura perché è una rolling-release}
    \alert{Errato!} Perché?
    \begin{itemize}
        \item Il lavoro di sicurezza sul software viene svolto \textbf{upstream}
        \item Arch Linux dispone di un ramo \textbf{Testing}
    \end{itemize}
\end{frame}

%-----------------------------------------------

\begin{frame}{Un aggiornamento può rompere il sistema}
    \alert{Errato!} Perché?
    \begin{itemize}
        \item Pacman non gestisce i file di configurazione e gli aggiornamenti da fare a questi ultimi: se qualcosa si rompe è colpa tua
        \item Quando un pacchetto viene aggiornato compaiono a video le eventuali istruzioni per adattare la configurazione all'ultimo aggiornamento 
        \item In poche parole? \textbf{Aggiornate frequentemente e leggete l'output!}
        \item Non vi fidate di voi stessi? Gli snapshot di sistema vengono in vostro aiuto
    \end{itemize}
\end{frame}

%-----------------------------------------------

\begin{frame}{Gli utenti di Arch sono spocchiosi}
    \alert{Errato!} Perché?
    \begin{itemize}
        \item Sono una risorsa per il mondo GNU/Linux ed ecosistema open-source
        \item Risolvono tutti i tuoi problemi con un semplice \textit{chroot}
        \item Sono dei veri appassionati del mondo Open Source
        \item Vogliono promuovere Arch e per questo ti daranno una mano
        \item \dots
    \end{itemize}
\end{frame}

%-----------------------------------------------

\begin{frame}{Arch vs Debian/Ubuntu}
    \textbf{Debian/Ubuntu}
    \begin{itemize}
        \item Periodic release
        \item Applicativi personalizzati per la distribuzione
        \item Installazione non estremamente personalizzata (nel caso di Ubuntu)
        \item Disponibili per molte architetture
    \end{itemize}

    \textbf{Arch}
    \begin{itemize}
        \item Rolling-release
        \item Pacchetti per come sono stati pensati dagli sviluppatori
        \item Installazione ad hoc
        \item Disponibile ufficialmente solo per x86\_64 ma esistono progetti per altre architetture
        \item Arch Build System(ports-like)
        \item AUR
    \end{itemize}
\end{frame}

\begin{frame}{Arch vs Debian/Ubuntu: in poche parole}
    \begin{itemize}
        \item Se vuoi un sistema funzionante alla svelta scegli una \textbf{Debian/Ubuntu}
        \item Se ti piace configurare e personalizzare il sistema in ogni suo aspetto e avere un'esperienza
        più vicina all'upstream scegli \textbf{Arch}
    \end{itemize}

    
    \begin{alertblock}{Tip}
        Se non hai mai utilizzato GNU/Linux fai prima esperienza con una Debian-based. Quando sentirai
        che la distribuzione che stai usando non ti dà il livello di libertà che cerchi passa ad Arch
    \end{alertblock}
\end{frame}

\begin{frame}{...e in ambito server?}
    \begin{itemize}
        \item Arch Linux non è particolarmente indicata per server di produzione
        \item In ambiente di produzione si tende a minimizzare il costo di manutenzione
        \item Nonostante ciò, vista la sua crescente popolarità, è possibile trovare Arch Linux tra le distribuzioni disponibili per VPS e server dedicati
    \end{itemize}
\end{frame}

