\begin{frame}{Arch Linux}
    \begin{figure}[h]
        \includegraphics[width=0.2\textwidth]{images/Archlinux-icon-crystal-64.png}
    \end{figure}
    \textbf{Arch Linux} è una distribuzione GNU/Linux\\

    \begin{itemize}
        \item Creata da Judd Vinet nel 2002
        \item Basata sulla filosofia \textit{KISS}
        \item Inizialmente ispirata a \textit{CRUX Linux}
    \end{itemize}
\end{frame}

%-----------------------------------------------

\begin{frame}{KISS}
    \textbf{KISS} è un acronimo per \textbf{Keep it simple, stupid} che suggerisce di evitare soluzioni  complesse
    poiché queste possono creare problemi. La semplicità è la cosa migliore.\\

    Arch Linux è basato su questa filosofia e per questa ragione non troverai nessun installer o software per la configurazione.\\
    Le configurazioni vengono gestite tramite la modifica di semplici file di testo(come papà UNIX vuole).
\end{frame}

%-----------------------------------------------

\begin{frame}{Pro Arch Linux}
    \begin{itemize}
        \item Rolling-release
        \item Installazione in base alle tue esigenze
        \item Sai tutto quello che ci sta dentro perché l'hai installato tu
        \item Ottima per esplorare GNU/Linux e capire meglio il suo funzionamento
        \item Arch Linux Wiki e Community
        \item AUR(questione di opinioni...)
        \item \dots
        \item ...usare "Btw I use Arch" come intercalare
    \end{itemize}

\end{frame}

%-----------------------------------------------

\begin{frame}{Contro Arch Linux}
    \begin{itemize}
        \item Ostica per i neofiti
        \item Installazione non immediata
        \item Se non si fa attenzione si rompe tutto (Soluzione: snapshot di sistema)
    \end{itemize}
\end{frame}

%-----------------------------------------------

\begin{frame}{4 falsi miti su Arch Linux}
    \begin{itemize}
        \item Arch Linux è estremamente complicato
        \item È meno stabile/sicura perché è una rolling-release
        \item Un aggiornamento può rompere il sistema
        \item Gli utenti di Arch sono spocchiosi
    \end{itemize}
\end{frame}

%-----------------------------------------------

\begin{frame}{Arch Linux è estremamente complicato}
    \alert{Errato!} Perché?
    \begin{itemize}
        \item La difficoltà risiede principalmente nella sua installazione e configurazione
        \item Dopo aver pagato il costo di apprendimento, si è più produttivi
        \item Installazioni e configurazioni articolate sono più semplici da realizzare
        
    \end{itemize}
\end{frame}

%-----------------------------------------------

\begin{frame}{È meno stabile/sicura perché è una rolling-release}
    \alert{Errato!} Perché?
    \begin{itemize}
        \item Il lavoro di sicurezza sul software viene svolto \textbf{upstream}
        \item Arch Linux dispone di un ramo \textbf{Testing}
        \item Vengono creati repository ex-novo per i nuovi pacchetti per non contaminare la stabilità dell'ecosistema.
    \end{itemize}
\end{frame}

%-----------------------------------------------

\begin{frame}{Un aggiornamento può rompere il sistema}
    \alert{Errato!} Perché?
    \begin{itemize}
        \item Pacman non gestisce i file di configurazione e gli aggiornamenti da fare a questi ultimi: se qualcosa si rompe è colpa tua
        \item Quando un pacchetto viene aggiornato compaiono a video le eventuali istruzioni per adattare la configurazione all'ultimo aggiornamento 
        \item In poche parole? \textbf{Leggete i log!}
        \item Non vi fidate di voi stessi? Gli snapshot di sistema vengono in vostro aiuto
    \end{itemize}
\end{frame}

%-----------------------------------------------

\begin{frame}{Gli utenti di Arch sono spocchiosi}
    \alert{Errato!} Perché?
    \begin{itemize}
        \item Sono una risorsa per il mondo GNU/Linux ed ecosistema open-source
        \item Risolvono tutti i tuoi problemi con un semplice \textit{chroot}
        \item Sono dei veri appassionati del mondo Open Source
        \item Vogliono promuovere Arch e per questo ti daranno una mano
        \item \dots
    \end{itemize}
\end{frame}


