\begin{frame}{Installazione minimale di Arch Linux}
    Effettueremo un'installazione con la seguente configurazione:
    \begin{itemize}
        \item \textbf{Desktop Environment}: GNOME
        \item \textbf{Display Server}: Wayland
        \item \textbf{Filesystem}: ext4 (singola partizione)
        \item \textbf{Cifratura disco}: Nessuna
    \end{itemize}
\end{frame}

%------------------------------------------------

\begin{frame}{Serve aiuto?}
    \begin{itemize}
        \item Installation Guide(\url{https://wiki.archlinux.org/index.php/Installation_guide})
        \item Arch Official Wiki(\url{https://wiki.archlinux.org/})
        \item Arch Official Forum(\url{https://bbs.archlinux.org/})
        \item ...
    \end{itemize}
\end{frame}

%------------------------------------------------

\begin{frame}{Creazione chiavetta USB}
    Prima di tutto sarà necessario creare una chiavetta USB avviabile.\\È possibile scaricare l'iso da \url{https://archlinux.org/download/}\\
    Per la creazione della chiavetta è possibile utilizzare:
    \begin{itemize}
        \item balenaEtcher(Mac, Windows, Linux)
        \item dd(Linux)
        \item Rufus(Windows)
        \item ...
    \end{itemize}

    \begin{block}{dd}
        \$ dd if=/percorso/a/archlinux.iso of=/dev/sd[x] bs=4M\\
    \end{block}
    \begin{alertblock}{Tip}
        Effettuate la primissima installazione su macchina virtuale per evitare il rischio di rompere tutto.
    \end{alertblock}
\end{frame}

%------------------------------------------------
\begin{frame}{Avvio da chiavetta USB}
    
\end{frame}